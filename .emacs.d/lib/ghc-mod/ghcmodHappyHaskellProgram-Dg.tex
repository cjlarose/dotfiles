% ghcmodHappyHaskellProgram-Dg.tex
\begin{hcarentry}[updated]{ghc-mod --- Happy Haskell Programming}
\report{Daniel Gr\"ober}%05/15
\status{open source, actively developed}
\makeheader

\texttt{ghc-mod} is both a backend program for enhancing editors and other kinds
of development environments with support for Haskell, and an Emacs package
providing the user facing functionality, internally called \texttt{ghc} for
historical reasons. Others have developed front ends for Vim, Atom and a few
other proprietary editors.

This summer's two month \texttt{ghc-mod} hacking session was mostly spent
(finally) getting a release supporting GHC 7.10 out the door as well as fixing
bugs and adding full support for the \textit{Stack} build tool.

Since the last report the \textit{haskell-ide} project has seen the light of day
(or rather been revived). There we are planning to adopt \texttt{ghc-mod} as a
core component to use its environment abstraction.

The \textit{haskell-ide} project itself (maybe soon to be called
\textit{ghc-ide-engine}) is aiming to be the central component of a unified
Haskell Tooling landscape.

\texttt{ghc-mod}'s mission statement remains the same but in the future it will
be but one, important, component in a larger ecosystem of Haskell Tools.

We are looking forward to \textit{haskell-ide} making the Haskell Tooling
landscape a lot less fragmented. However until this project produces meaningful
results life goes on and \texttt{ghc-mod} needs to be maintained.

Right now \texttt{ghc-mod} has only one core developer and only a handful of
occasional contributors. If \textit{you} want to help make Haskell development
even more fun come and join us!

\FurtherReading
  \url{https://github.com/kazu-yamamoto/ghc-mod}
\end{hcarentry}
